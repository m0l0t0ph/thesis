% globale Parameter (Papierformat, Schriftgroeße, Trennlinien, Doppelseitig)
\documentclass[a4paper, 11pt, headsepline,footsepline,twoside,abstract]{scrbook}


%%% Verwendete Pakete %%%

\usepackage{url}
\usepackage{geometry} % Seitenraender
\usepackage{fancyhdr} % Schoenere Kopf/Fußzeilen
\usepackage[utf8]{inputenc} % Kodierung
\usepackage[ngerman]{babel} % Sprache
\usepackage{graphicx}  % Bildchen
\usepackage{float}  % Bildchen2
\usepackage{setspace} % fuer Zeilenabstand
\usepackage[T1]{fontenc} % fuer Schriftart
%\usepackage{cite} % fuer Zitate
\usepackage[numbers,square]{natbib} % Package für Zitierstil
\usepackage{pgfplots} % Plots
\usepackage[justification=RaggedRight, singlelinecheck=false, margin=1.5cm]{caption}  % huebschere Captions
\usepackage[version=3]{mhchem} % chemische Formeln
\usepackage{booktabs} % schoenere Tabellen
\usepackage{multirow} % s.o.
\usepackage{ifthen} % s.o.
\usepackage{subfigure} % Grafiken nebeneinander darstellen
\usepackage{textcomp} %Euro-Zeichen
\usepackage{siunitx} % Einheiten korrekt anzeigen
%\sisetup{
% locale = DE ,
% per-mode = symbol,
% separate-uncertainty %+- Notation
%}
%\DeclareSIUnit{\masspercent}{m\%}
%\newcommand{\euro}{\,\texteuro\ } %Euro einfuegen

%Schriftart aendern
\newcommand{\changefont}[3]{
\fontfamily{#1} \fontseries{#2} \fontshape{#3} \selectfont}
\changefont{ptm}{m}{n}

% Zeilenabstand 1,5
\onehalfspacing

% weniger breite Seitenraender
\geometry{a4paper,left=28mm,right=28mm, top=32mm, bottom=30mm} 

% Keine Einrueckung bei neuen Absätzen
%\setlength{\parindent}{0pt}
%\setlength{\parskip}{\baselineskip}

% Bissel Tabellenmagie
\newcommand{\forloop}[5][1]{%
\setcounter{#2}{#3}%
\ifthenelse{#4}{#5\addtocounter{#2}{#1}%
\forloop[#1]{#2}{\value{#2}}{#4}{#5}}%
{}}

\newcounter{crcounter}

\newcommand{\compensaterule}[1]{%
\forloop{crcounter}{1}{\value{crcounter} < #1}%
{\vspace*{-\aboverulesep}\vspace*{-\belowrulesep}}}

\newcommand{\multirowbt}[3]{\multirow{#1}{#2}%
{\compensaterule{#1}#3}}

%%% Kopf und Fußzeilendesigns %%%

% normaler Style
\fancypagestyle{normal}
{
\fancyhead[EL]{\thepage}
\fancyhead[ER]{\leftmark}
\fancyhead[OL] {\rightmark}
\fancyhead[OR]{\thepage}
\fancyfoot[OL]{\parbox[b][0.04\columnwidth][t]{0.5\textwidth}{\raggedright Masterarbeit, Christoph Gielisch}}
\fancyfoot[OR]{
	\includegraphics[height=0.05\columnwidth]{images/KIT_LOGO.png}
	%\caption{}
	\label{img:grafik-dummy}
	}
\fancyfoot[OC,EC]{}
\fancyfoot[EL]{
	\includegraphics[height=0.05\columnwidth]{images/KIT_LOGO.png}
	%\caption{}
	\label{img:grafik-dummy}
	}
\fancyfoot[ER]{\parbox[b][0.04\columnwidth][t]{0.5\textwidth}{\raggedleft  Thema Masterarbeit}}
\renewcommand{\headrulewidth}{0.5pt}
\renewcommand{\footrulewidth}{0.5pt}
}

% Kapitelanfaenge bekommen ein extra Kopf/Fußzeilendesign
\fancypagestyle{plain}
{
\fancyhf{} % clear all header and footer fields
\fancyfoot[R]{\parbox[b][0.04\columnwidth][t]{0.5\textwidth}{\raggedleft \thepage}}
\fancyfoot[L]{
	\includegraphics[height=0.05\columnwidth]{images/leerpixel.png}
	%\caption{}
	\label{img:grafik-dummy}
	}
\renewcommand{\headrulewidth}{0pt}
\renewcommand{\footrulewidth}{0.5pt}
}

% Die Inhaltsangabe auch
\fancypagestyle{toc}
{
\fancyhf{} % clear all header and footer fields
\fancyfoot[OR]{\parbox[b][0.04\columnwidth][t]{0.5\textwidth}{\raggedleft \thepage}}
\fancyfoot[OL]{
	\includegraphics[height=0.05\columnwidth]{images/leerpixel.png}
	%\caption{}
	\label{img:grafik-dummy}
	}
\fancyfoot[EL]{\parbox[b][0.04\columnwidth][t]{0.5\textwidth}{\raggedright \thepage}}
\fancyfoot[ER]{
	\includegraphics[height=0.05\columnwidth]{images/leerpixel.png}
	%\caption{}
	\label{img:grafik-dummy}
	}
\renewcommand{\headrulewidth}{0pt}
\renewcommand{\footrulewidth}{0.5pt}
}

% Festlegung Art der Zitierung
%\bibliographystyle{unsrtdin}

%%%%%%%%%%%%%%%%%%% hier beginnt das Dokument %%%%%%%%%%%%%%%%%%%%
\begin{document}

\thispagestyle{empty}
\begin{center}
%\Large{Karlsruher Institut für Technologie}\\
\end{center}

\begin{figure}
	\centering
	\includegraphics[width=0.5\columnwidth]{images/KIT_LOGO.png}
	%\caption{}
	\label{img:grafik-dummy}
\end{figure}

\begin{center}
\textbf{\huge{ Aufbau einer in-situ Li$^7$-NMR\\[0.4cm]Batterietestzelle}}
\end{center}
\begin{center}
\large{}
\end{center}
\begin{center}
\textbf{\Large{}}
\end{center}
\begin{center}
\large{Von der Fakultät für Wirtschaftswissenschaften des \\ Karlsruher Instituts für Technologie genehmigte }
\end{center}
\begin{verbatim}

\end{verbatim}
\begin{center}
\textbf{\LARGE{Masterarbeit}}
\end{center}
\begin{center}
am
\end{center}
\begin{center}
\textbf{\Large{Institut für Angewandte Materialien - Keramische Werkstoffe und Technologien (IAM-KWT)}}
\end{center}
\begin{center}
von
\end{center}
\begin{center}
\Large{Christoph Gielisch}
\end{center}
\begin{verbatim}

\end{verbatim}
\begin{center}
30. September 2015
\end{center}
\begin{verbatim}

\end{verbatim}
\begin{center}
\textbf{Referent:} \\ Prof. Dr-Ing. Volker Schulze \\
\textbf{Koreferent:} \\ Prof. Dr. Michael J. Hoffmann\\
\textbf{Betreuer:} \\ Dr. Claudia Bucharsky \\ 
Dr.-Ing. Günter Schell \\
\end{center}
\newpage
\cleardoubleemptypage
% Eidesstattliche Erklärung
\setcounter{page}{1}
\pagenumbering{Roman}
\textbf{\Large{Eidesstattliche Erklärung}}
\\\\
Hiermit erkläre ich, diese Arbeit selbstständig und ohne fremde Hilfe verfasst zu haben. Es wurden nur die in der Arbeit ausdrücklich benannten Quellen und Hilfsmittel benutzt. Wörtlich oder sinngemäß übernommenes Gedankengut ist als solches gekennzeichnet.
\\\\
Karlsruhe, den 30.09.2015
\\\\
\\\\
\\
(Christoph Gielisch) 
 
\newpage

% Eidesstattliche Erklärung2
\setcounter{page}{1}
\pagenumbering{Roman}
\textbf{\Large{Zusammenfassung}}
\\\\
Es ist möglich die Oberfläche von Aluminiumfolien für den Einsatz als Stromkollektor in Batterien gezielt zu strukturieren. Es ist gelungen, strukturierte Folien zu funktionalen \ce{LiCoO2}-Elektroden zu verarbeiten und in Testzellen einzusetzen. Mit Hilfe einer Impedanzspektroskopie konnte der Widerstand zwischen Stromkollektor und Aktivmaterial untersucht werden. Dabei zeigte eine sandgestrahlte Aluminiumoberfläche ein verbessertes Verhalten gegenüber einer unbearbeiteten Oberfläche. Für einen wirtschaftlichen Einsatz der Technologie sind allerdings weitere Untersuchungen hinsichtlich der Fertigung mit Roll-to-Roll-Anlagen notwendig.
% Inhaltsverzeichnis
%\KOMAoption{open}{left} 
\pagestyle{toc}
\renewcommand*{\chapterpagestyle}{toc} % Die erste Seite des TOC ist auch ein Kapitelanfang
\tableofcontents
\KOMAoption{open}{right} 
\newpage
\cleardoubleemptypage
\pagestyle{normal}
\renewcommand*{\chapterpagestyle}{plain}
\setcounter{page}{1}
\pagenumbering{arabic}
\chapter{Einleitung}
Batterien und Kondensatoren sind bereits heutzutage in den meisten elektrischen Geräten verbaut. Im Rahmen des Klimaschutzes ist in Zukunft ein stärkerer Einsatz von elektrischen Komponenten als Substitut für bisher mittels fossiler Brennstoffe betriebener Geräte und Maschinen zu erwarten. In den letzten Jahren konnte die Leistungsfähigkeit von Batterien schon beträchtlich gesteigert werden, wie Abbildung 1.1 verdeutlicht. Im Rahmen dieser Arbeit soll die Möglichkeit einer weiteren Verbesserung durch Mikrostrukturierung der Elektroden sowie deren wirtschaftliche Machbarkeit überprüft werden.
\section{Motivation}
\section{Zielsetzung}
\section{Aufbau der Arbeit}
\chapter{Grundlagen}
\section{Batteriezellen}
\subsection{Elektrochemischer Vorgang}
\subsection{Aufbau}
\subsection{Elektroden}
\subsection{Elektrolyten}
\section{Analysemethoden}
\subsection{Röntgendiffraktion}
\subsection{Impedanzspektroskopie}
\subsection{NMR-Spektroskopie}
Die Grundlage der Kernspinresonanzspektroskopie (nuclear magnetic resonance spectroscopy, NMR-Spektroskopie) wurde zum Jahreswechsel 1945/1946 von zwei amerikanischen Forschungsgruppen unabhängig voneinander entwickelt. Felix Bloch und Edward M. Purcell wurden dafür 1952 mit dem Nobelpreis in Physik ausgezeichnet.
\\\\
\subsubsection{Physikalische Grundlagen der NMR-Spektroskopie}
Die NMR-Spektroskopie nutzt die magnetischen Eigenschaften von Atomkernen und ihren Umgebungen aus, um Aussagen über Zusammensetzungen und Bindungen von Stoffen treffen zu können.
\\\\
Elektronen, Neutronen und Protonen besitzen eine Eigenrotation, den Spin s. Der Spin eines Atomkerns setzt sich aus den Spins der einzelnen Protonen und Neutronen innerhalb des Kerns zusammen. Spins sind gequantelt und können daher nur gewisse diskrete Zustände annehmen. Dies gilt auch für den resultierenden Gesamtspin des Atomkerns. Die möglichen Zustände des Kernspins eines spezifischen Isotops können beschrieben werden über seine Spinquantenzahl I. Es existieren folgende magnetische Spinquantenzahlen m, welche die möglichen Orientierungen des Spins beschreiben:
\begin{equation}
m_I = I, I-1, I-2, ..., -I
\end{equation}
Die Gesamtzahl an möglichen Zuständen entspricht daher der Summe von 2I+1. Das Li$^7$ besitzt die Spinquantenzahl I=$\frac{3}{2}$. Es gilt daher:
\begin{equation}
m_{I=\frac{3}{2}} = \frac{3}{2}, \frac{1}{2}, -\frac{1}{2}, -\frac{3}{2}
\end{equation}
Sind in einem Atomkern die Anzahl an Protonen und Neutronen beide gerade, so gilt für die Spinquantenzahl I=0. Ein solcher Nukleus besitzt keinen Kernspin und kann daher nicht mittels NMR-Spektroskopie untersucht werden.
\\\\
Ein Atomkern besitzt eine Ladung. Wenn diese durch einen Kernspin bewegt wird, so besitzt der Nukleus ein magnetisches Moment $\mu$ in Abhängigkeit zum Zustand des Spins. Der Zusammenhang zwischen einem Drehmoment P und dem magnetischen Moment kann allgemein über das gyromagnetische Verhältnis $\gamma$ beschrieben werden: 
\begin{equation}
\mu = \gamma P
\end{equation}
Das Drehmoment des Kerns in Richtung z eines frei gewählten kartesischen Koordinatensystems entspricht dabei seiner magnetischen Spinquantenzahl multipliziert mit dem reduzierten Planckschen Wirkungsquantum:
\begin{equation}
P_z = m_I \hbar
\end{equation}
Das magnetische Moment kann also beschrieben werden mit:
\begin{equation}
\mu_z = \gamma m_I \hbar
\end{equation}
Keiner dieser möglichen Spinzustände ist energetisch günstiger als die anderen. Die Zustände liegen daher degeneriert vor. Dies kann allerdings durch das Anlegen eines starken äußeren Magnetfeldes B$_0$ in positiver z-Richtung beeinflusst werden. Es bilden sich verschiedene Energieniveaus für die unterschiedlichen Spinzustände aus. Die Energiedifferenz zwischen den Zuständen ist dabei proportional zur Stärke des angelegten äußeren Magnetfelds. Die Spins richten sich entlang der Achse aus. Die Energiediffernzen sind dabei für jeden Kern, der einen Spin besitzt, charakteristisch und können mit einer Frequenz in Abhängigkeit zur Stärke des äußeren Magnetfelds beschrieben werden. Diese Frequenz wird als Larmor-Frequenz bezeichnet und kann auch als Präzession des Kerns beschrieben werden.
\subsubsection{Aufbau eines NMR-Spektrometers}
\subsubsection{Betriebsmodus}

\begin{figure}[h]
	\centering
	\includegraphics[width=0.7\columnwidth]{images/Schema_Zelle.jpg}
	\caption{Schema der Testzelle; 
			1: Plastikkappe,
			2: Dichtungsring,
			3: Edelstahlfeder,
			4: Kathode,
			5: Anode,
			6: Plastikkappe,
			7: Glaszylinder,
			8: Kontaktierplättchen,
			9: Separator,
			10: Edelstahlstopfen.
			}
	\label{schema_zelle}
\end{figure}
\subsubsection{Herstellung der fertigen Testzelle}
Die Testzellen müssen unter Schutzgasatmosphäre in einer Handschuhbox (siehe Abbildung \ref{handschuhbox}) gefertigt werden. Dies ist nötig, da sowohl das rein metallische Lithium, aber insbesondere auch der zum Einsatz kommende Flüssigelektrolyt mit dem Luftsauerstoff und der Luftfeuchtigkeit reagieren können. Als Flüssigelektrolyt kommt dabei das kommerzielle Produkt LP30 der Firma Merck zum Einsatz. Dieser besteht aus einem organischen Lösungsmittel, in dem verschiedene Lithiumsalze als Flourverbindungen gelöst sind. In Kontakt mit Luftfeuchtigkeit reagiert dieser und es entsteht hochätzende Flusssäure. 
\\\\
Für den Aufbau einer Testzelle wird ein Stempel in einen Glaszylinder eingeführt und mit einer Kappe fixiert. Anschließend kann die Zelle von der anderen Seite des Zylinders befüllt werden. Dafür wird zunächst ein Edelstahlplättchen und die Kathodenprobe eingesetzt. Darauf werden zwei Lagen Separator gelegt. Dadurch hat man die Sicherheit, dass selbst bei einem lokalen Defekt eines Separators immer noch das Verhindern des Kurzschlusses gewährleistet ist. Die Separatoren werden dann mit \SI{160}{\micro\litre} Elektrolyt benetzt. Nun kann die Lithiumanode und das obere Edelstahlplättchen abgelegt werden. Darauf wird die Feder gestellt und mit dem oberen Stempel und der oberen Kappe komprimiert sowie die Testzelle verschlossen.
\begin{figure}[h]
	\centering
	\includegraphics[width=1.0\columnwidth]{images/handschuhbox.jpg}
	\caption{Handschuhbox der Firma MBraun}
	\label{handschuhbox}
\end{figure}
\chapter{Methodik}
\section{Pulverherstellung}
\subsection{LATP}
\subsection{\ce{LiTi4O5}}
\subsection{Weitere verwendete Pulver}
\section{Elektrodenherstellung}
\subsection{Herstellung verschiedener Elektrodenslurries}
\subsection{Foliengießen der Elektroden}
\subsection{Herstellung eines Elektrode-Elektrolyt-Gefüges}
\section{Konstruktion der in-situ-Testzelle}
\subsection{Anforderungen}
\subsection{Planung}
\subsection{Angefertigte Teile}
\chapter{Ergebnisse}
\section{Pulveranalyse}
\subsection{XRD-Analyse}
\subsection{MAS-NMR-Analyse}
\section{Batterietests}
\subsection{Ladekennlinien}
\subsection{Impedanzmessungen}
\subsection{NMR-Messungen}
\chapter{Diskussion}
\section{Anwendbarkeit NMR}
\section{Vergleich Batterien}
\chapter{Fazit}
Abschließend werden die wichtigsten Erkenntnisse dieser Arbeit noch einmal kurz zusammengefasst und mögliche Verbesserungs- und Veränderungsvorschläge in einem Ausblick gegeben.
\section{Zusammenfassung}
Die Herstellung gezielt strukturierter Aluminiumfolien mit mikrostrukturierter und sandgestrahlter Oberfläche für den Einsatz in Batteriezellen ist möglich. Diese konnten zu Elektrodenplättchen weiterverarbeitet werden. Die Elektrodenplättchen wurden erfolgreich in Batterietestzellen eingebaut und konnten über mehrere Zyklen hinweg getestet werden.
\\\\
Mittels Impedanzspektroskopie konnten die Proben anschließend auf ihren Wiederstand zwischen Aktivmaterial und Aluminiumfolie hin untersucht werden. Dabei wurde festgestellt, dass nur die sandgestrahlte Oberfläche ein verbessertes Verlustverhalten aufwies. 
\\\\
Eine wirtschaftliche Produktion von Batterien mit strukturierten Stromkollektorfolien ist denkbar. Es wird allerdings noch eine genauere, quantitative Untersuchung des Verbesserungspotentials der Technologie benötigt, um konkrete Berechnungen durchführen zu können. Weiterhin muss gezeigt werden, dass die Strukturierung auch mittels Roll-to-Roll-Anlagen in ausreichender Geschwindigkeit realisierbar ist.
\section{Ausblick}
Auf Grundlage dieser Arbeit gibt es verschiedene Fragestellungen, welche in Zukunft weiter untersucht werden können. Die verwendete Mikrostruktur kann weiter verbessert werden. Im Vergleich zur Höhe der Aktivmaterialschicht auf kommerziellen Zellen von $100 - 200\mu m$ ist die Struktur noch zu flach, um tief genug in das Material hinein zu reichen. Eine höhere und dafür auch breitere Struktur könnte hier bessere Ergebnisse erzielen.
\\\\
Das Aktivmaterial selbst kann auch auf die Mikrostruktur hin optimiert werden. Dabei wäre beispielsweise denkbar durch die kürzeren Wege den Anteil an Leitruß zu verringern und dadurch die Energiedichte weiter zu verbessern. Auch ein Kalendrieren der Probe sollte in Betracht gezogen werden, um einen besseren Kontakt zwischen Aktivmaterial und Stromkollektor zu gewährleisten. Eine genauere Analyse mit Hilfe eines Rasterelektronenmikroskops kann hier genaueren Aufschluss über den Kontakt zwischen den Schichten geben.
\\\\
Für die Batterietests sind weiterführende Experimente zu empfehlen. Dabei sollte vor allem die Anzahl an Proben angehoben werden und auch Langzeittests in Betracht gezogen werden. Die Impedanzmessungen können durch bessere Proben und gezieltere Beeinflussung der Umgebungsfaktoren wie verschiedene Ladungszustände und Temperaturen besser analysiert werden.
\\\\
Aus Zeitgründen konnte der Test des strukturierten Aluminiums als Material für Kondensatoren nicht realisiert werden. Auch die Herstellung der Folie über Roll-to-Roll-Anlagen muss noch genauer erforscht werden. Erst dann sind weiterführende Aussagen über eine auch kommerzielle Eignung der Technik möglich.
% der Anhang
\renewcommand{\thesection}{\Alph{section}}
%\appendix
%\addchap{Anhang}
%\section{Parametereinstellungen}
%\section{Vergleichsbilder}

% das Abbildungsverzeichnis
\cleardoublepage
% \phantomsection
\addcontentsline{toc}{chapter}{Abbildungsverzeichnis}
\listoffigures

 % das Literaturverzeichnis
\cleardoublepage
% \phantomsection
\addcontentsline{toc}{chapter}{Literaturverzeichnis}
\bibliographystyle{unsrtdin}
\bibliography{literatur} 

% das ist wohl jetzt das Ende des Dokumentes
\end{document}